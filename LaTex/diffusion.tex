\documentclass[resume]{subfiles}


\begin{document}
\section{Équation de diffusion}
$$\boxed{u_{t}=u_{xx}}$$
Plus difficile à résoudre que l'équation d'ondes
\subsection{Principe du maximum}
Valeur maximale de $u(x,t)$ atteinte à $t=0$ ou sur les côtés ($x=0$ ou $x=l$). Pareil pour la valeur minimale
\subsection{Résolution}
\begin{enumerate}
\item Résoudre l'équation pour une solution $\phi(x)$ particulière
\item Construire la solution générale
\end{enumerate}
\subsection{Propriétés}
\begin{enumerate}
\item Une \textbf{translation} de la solution est aussi une solution
$$u(x-n,t)\equiv u(x,t)$$
\item \textbf{Dérivée} d'une solution est aussi une solution
$$u_t\equiv u_x\equiv u_{xx}\equiv u$$
\item Une \textbf{combinaison linéaire} de solutions est une solution
\item Une \textbf{intégrale} est aussi une solution
$$\int S(x-n,t)g(y)dy\equiv u(x,t)$$
\item Une solution \textbf{dilatée} est aussi une solution
$$u(\sqrt{a}x,at)\equiv u(x,t)$$
\end{enumerate}
\subsection{Résolution sans conditions aux bords}
On résout le problème simplifié avec
$$Q(x,0)=\begin{cases}1 & x>0\\0 & x<0\end{cases}$$
$$Q(x,t)=g(p)\qquad p=\frac{x-y}{\sqrt{4kt}}$$
Solution générale :
$$u(x,t)=\frac{1}{2\sqrt{\pi kt}}\int_{-\infty}^{\infty}e^{-\frac{(x-y)^2}{4kt}}\phi(y)dy$$
\begin{enumerate}
\item Remplacer la condition initiale $\phi(x)$
\item Développer l'intégrale et effectuer un changement de variable si nécessaire (voir \ref{sec_autres})
\item Exprimer en fonction de erf(...) si c'est nécessaire
\end{enumerate}
Si nécessaire, on utilise la fonction d'erreur
$$\text{erf}(x)=\frac{2}{\sqrt{\pi}}\int_{0}^{x}e^{-p^2}dp$$
$$\text{erf}(x)=-\text{erf}(-x)\qquad (\text{impaire})$$
Si $\phi(y)=e^{...}$ alors on peut utiliser la fonction suivante (à adapter) pour mettre tous les $y$ dans le $()^2$
$$(y+2kt-x)^2=y^2+4k^2t^2+x^2+4kty-2xy-4ktx$$
\subsubsection{Notes}
$$\int_{-\infty}^{\infty}e^{-p^2}dp=\sqrt{\pi}$$
Si on a deux intégrales (chacune avec un $\phi(y)$ différent, par exemple un $\phi$ par morceaux), alors on fait deux changements de variables différents : une fois $p=\frac{x-y}{\sqrt{4kt}}$ et une fois $p=\frac{y-x}{\sqrt{4kt}}$
\subsection{Résolution avec conditions aux bords}
Par séparation de variables on a 
$$u(x,t)=X(x)T(t)$$
$$\frac{T'}{kT}=\frac{X''}{X}=-\lambda$$
$$\begin{cases} T(t)=Ae^{-\lambda kt}\\
X(x)=B\cos(\beta x)+C\sin(\beta x)\end{cases}\qquad \lambda=\beta^2$$
Résoudre en appliquant les conditions aux bords à l'équation ci-dessus.\\
Si il est possible d'exprimer $u_{n=0}(x,t)$ avec une constante, on la nomme $\frac{A_0}{2}$





\end{document}