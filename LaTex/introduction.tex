\documentclass[resume]{subfiles}


\begin{document}
\section{Introduction}
$$\boxed{\frac{\partial u}{\partial x}=u_x}$$
$$\frac{\partial^2 u}{\partial \textcolor{RoyalBlue}{x}\partial \textcolor{OrangeRed}{y}}=\frac{\partial^2 u}{\partial \textcolor{OrangeRed}{y}\partial \textcolor{RoyalBlue}{x}}$$
\subsection{Dérivée}
$$u'(x)=\lim_{h\to 0}\frac{u(x+h)-u(x)}{h}=\lim_{h\to 0}\frac{u(x)-u(x-h)}{h}$$
\subsection{EDO du premier ordre}
$$\boxed{\frac{dy}{dt}=ky\longrightarrow y=Ce^{kt}}$$
\subsection{EDP du premier ordre}
\begin{multline*}
F\Big(x,y,u(x,y),u_x(x,y),u_y(x,y)\Big)\\=F\Big(x,y,u,u_x,u_y\Big)=0
\end{multline*}
\subsection{EDP du deuxième ordre}
$$F\Big(x,y,u,u_x,u_y,u_{xx},u_{xy},u_{yy}\Big)=0$$
$$Au_{xx}+Bu_{xy}+Cu_{yy}+Du_x+Eu_y+Fu=G$$
\paragraph{Parabolique} : $B^2-4AC=0$
\paragraph{Hyperbolique} : $B^2-4AC>0$
\paragraph{Elliptique} : $B^2-4AC<0$
\subsection{Opérateurs}
\paragraph{Linéarité}
$$\mathcal{L}(u+v)=\mathcal{L}u+\mathcal{L}v\quad \text{ et }\quad \mathcal{L}(cu)=c\mathcal{L}u$$
\hfill \textcolor{RoyalBlue}{linéaire}\hfill \textcolor{OrangeRed}{non linéaire}\hfill \
$$\textcolor{RoyalBlue}{u_x+u_y=0}\qquad \textcolor{RoyalBlue}{u_x+yu_y=0}\qquad \textcolor{OrangeRed}{u_x+uu_y=0}$$
$$\textcolor{RoyalBlue}{u_{xx}+u_{yy}=0}\quad \textcolor{OrangeRed}{u_{tt}-u_{xx}+u^3=0}\quad \textcolor{OrangeRed}{u_t+uu_x+u_{xxx}=0}$$
\paragraph{Équation linéaire homogène} $\mathcal{L}u=0$
\paragraph{Équation linéaire non-homogène} $\mathcal{L}u=g$
$$u_x+u_y+1=0\longrightarrow \text{ inhomogène}$$
solution homogène + solution inhomogène = solution inhomogène
$$\boxed{a(x,y)u_x+b(x,y)u_y=0\longrightarrow \frac{dy}{dx}=\frac{b(x,y)}{a(x,y)}}$$
Puis faire
$$\underbrace{\int \frac{dy}{dx}dx}_{y}=\underbrace{\int \frac{b(x,y)}{a(x,y)}dx}_{\cdots + c}$$
Combinaison linéaire de plusieurs solutions est aussi une solution
\subsection{Conditions initiales}
$$u(x,t_0)=\phi(x)$$
OU
$$u(x,t_0)=\phi(x)\qquad u_t(x,t_0)=\psi(x)$$
\subsection{Conditions aux bords}
\paragraph{Dirichlet} : $u$ est spécifié
\paragraph{Neumann} : $\frac{\partial u}{\partial n}$ est spécifié
\paragraph{Robin} : $\frac{\partial u}{\partial n}+au$ est spécifié
\subsection{Problèmes bien posés}
Les problèmes bien posés (au sens d'Hadamard) sont constitués d'une EDP dans un domaine et avec les propriétés suivantes :
\paragraph{Existence} : il existe au moins une solution $u(x,t)$ qui satisfait toutes les conditions
\paragraph{Unicité} : il existe au plus une solution
\paragraph{Stabilité} : La solution unique $u(x,t)$ dépende de manière stable des données (peu de changement $\to$ peu de variation)
\subsection{Exemples}
\begin{enumerate}
\item 
$$au_x+bu_y=0\qquad u(x,y)=f(bx-ay)$$
Avec $bx-ay=c$ les droites caractéristiques
\item 
$$u_t+cu_x=0$$
Au temps $t+h$, déplacement de $c\cdot h$
\item 
$$u_{xx}=0\xrightarrow{\int dx} u_x=f(y)\xrightarrow{\int dx} u=g(y)+xf(y)$$
$$u(x,y)=f(y)x+g(y)$$
\item \begin{small}$$u_{xx}+u=0\rightarrow u(x,y)=f(y)\cos(x)+g(y)\sin(x)$$\end{small}
\item $$u_{xy}=0\longrightarrow u(x,y)=f(y)+g(x)$$
A noter que $f(y)$ et $g(x)$ sont les intégrales de fonctions intermédiaires.
\item $$u_x+yu_y=0\longrightarrow u(x,y)=f(e^{-x}y)$$

\end{enumerate}




\end{document}