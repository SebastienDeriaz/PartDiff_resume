\documentclass[resume]{subfiles}


\begin{document}
\section{Éléments finis}
\subsection{Forme forte}
$$-u''(x)=f(x)$$
\subsection{Forme faible / variationnelle (Ritz-Galerkin)}
$$-u''(x)v(x)=f(x)v(x)$$
On multiplie des deux côtés par une fonction $v(x)$ qui respecte
$$v(0)=v(l)=0$$
$$\int_{0}^{l}-u''(x)v(x)dx=\int_{0}^{l}f(x)v(x)$$
\subsubsection{Exemple}
Avec $-u''(x)=x^2$. On aura un problème de la forme
$$\boxed{A_hc=b_h}$$
Avec les $c$ qui correspondent au poids de chaque fonction de base.
\paragraph{Calcul de $A$ (matrice de rigidité)}
$$\boxed{a_{ij}=\int_{0}^{L}N_i'(x)N_j'(x)dx}$$
\paragraph{Calcul de $b$}
$\vec{b}$ est le reste de l'équation (partie droite)
$$\int_{0}^{L}-u''(x)v(x)=\int_{0}^{1}f(x)v(x)$$
La plupart du temps on aura
$$\boxed{b_i=\int_{0}^{l}f(x)N_i(x)}$$
\subsection{Maillage}
Il ne doit pas y avoir de chevauchement d'éléments ou de points qui ne sont pas connectés ensembles.
\end{document}