\documentclass[resume]{subfiles}


\begin{document}
\section{Séries de Fourier}
\subsection{Séries de Fourier en sinus}
$$\boxed{\phi(x)=\sum_{n=1}^{\infty}A_n\sin\left(\frac{n\pi}{l}x\right)}$$
$$\boxed{A_n=\frac{2}{l}\int_{0}^{l}\phi(x)\sin\left(\frac{n\pi x}{l}\right)dx}$$
\subsection{Séries de Fourier en cosinus}
$$\boxed{\phi(x)=\frac{A_0}{2}+\sum_{n=1}^{\infty}A_n\cos\left(\frac{n\pi}{l}x\right)}$$
$$\boxed{A_n=\frac{2}{l}\int_{0}^{l}\phi(x)\cos\left(\frac{n\pi x}{l}\right)dx}$$
Le $1/2$ dans la série pour $A_0$ vient de la\\
\textcolor{OrangeRed}{Important} : Si la fonction $\phi(x)$ est paire, on peut se concentrer sur la moitié uniquement (et faire $\frac{1}{l}$ au lieu de $\frac{2}{l}$, la valeur de $l$ est ce nouvel intervalle). Ceci permet de beaucoup simplifier le problème.
\subsection{Séries de Fourier}
Sur $]-l.l[$
$$\boxed{\phi(x)=\frac{A_0}{2}+\sum_{n=1}^{\infty}\left(A_n\cos\left(\frac{n\pi x}{l}\right)+B_n\sin\left(\frac{n\pi x}{l}\right)\right)}$$

\begin{align*}
A_n &= \frac{1}{l}\int_{0}^{l}\phi(x)\cos\left(\frac{n\pi}{l}x\right)dx\\
B_n &=\frac{1}{l}\int_{0}^{l}\phi(x)\sin\left(\frac{n\pi}{l}x\right)
\end{align*}





\end{document}