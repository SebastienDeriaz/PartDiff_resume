\documentclass[resume]{subfiles}


\begin{document}
\section{Équation d'onde}
$$\boxed{u_{tt}=c^2u_{xx}}$$
\subsection{1D}
Modèle ressorts-masses
$$F_\text{newton}=ma(t)=m\frac{\partial^2}{\partial t^2}u(x,t)$$
et
$$F_\text{hooke}=k\left(u(x+h),t)-u(x,t)\right)-k\left(u(x,t)-u(x-h,t)\right)$$
$$F_\text{newton}=F_\text{hooke}$$
Avec $N\to\infty$ et donc $h\to 0$ ($L=Nh$)
$$\frac{\partial^2 u(x,t)}{\partial t^2}=\frac{KL^2}{M}\frac{\partial^2 u(x,t)}{\partial x^2}$$
Solution générale :
$$\boxed{u(x,t)=f(x+ct)+g(x-ct)}$$
avec $f$ et $g$ des fonctions quelconques à une seule variable
\subsubsection{Propriétés}
Deux familles de droites caractéristiques $x\pm ct=\text{constante}$. Somme de deux fonctions : $g(x-ct)$ qui va à droite et $f(x+ct)$ qui va à gauche. La vitesse est $c$.
\subsubsection{Conditions initiales, pas de conditions aux bords}
$$u_{tt}=c^2u_{xx}\qquad -\infty<x<\infty$$
$$u(x,0)=\phi(x)\qquad u_t(x,0)=\psi(x)$$
$$u(x,t)=\frac{1}{2}\left(\phi(x+ct)+\phi(x-ct)\right)+\frac{1}{2c}\int_{x-ct}^{x+ct}\psi(s)ds$$
\subsection{Avec conditions aux bords}
$$u(0,t)=0\qquad u(l,t)=0$$
$$u(x,0)=\phi(x)\qquad u_t(x,0)=\psi(x)$$
La solution est de la forme 
$$u(x,t)=X(x)T(t)$$
$$\begin{cases}X(x)&=C\cos(\beta x)+D\sin(\beta x)\\T(t) &= A\cos(\beta ct)+B\sin(\beta ct)\end{cases}$$
$\lambda$ est une constante tel que $\lambda=\beta^2\qquad \beta>0$
\end{document}